\documentclass[a4paper]{book}
\usepackage[utf8]{vietnam}
\title{Luận văn tốt nghiệp}
\author{Nguyễn Thái Ngọc Duy}
\begin{document}
\maketitle

\chapter{Giới thiệu}
\label{cha:intro}


\chapter{Background}
\label{cha:background}

\section{Tổng quan về lỗi chính tả}

Văn bản nhập vào thường hay bị lỗi. Các nguyên nhân gây ra lỗi là:
\begin{itemize}
\item Do quá trình nhập bị sai (gõ nhầm, lỗi OCR ...)
\item Do lầm lẫn giữa cách đọc và cách viết
\item Do hiểu sai dẫn đến viết sai.
\end{itemize}

Luận văn này chỉ giải quyết hai lỗi đầu.

\section{Xử lý lỗi chính tả}

Có hai hướng chính.
\begin{itemize}
\item Tìm lỗi, đề nghị cách sửa chữa.
\item Tự động sửa lỗi.
\end{itemize}

Tìm lỗi dựa chủ yếu trên từ điển. So từng từ với từ điển, những từ
không có trong từ điển là những từ có khả năng bị lỗi. Sau đó dựa trên
một số heuristic hoặc các độ đo để tìm ra từ gần đúng với từ đó, làm
từ đề nghị.

Tự động sửa lỗi chủ yếu dựa trên một tập các từ hay bị lỗi (then-than,
there-their \ldots). Sử dụng ngữ cảnh xung quanh để xác định từ đúng
hay sai. Cách này chỉ áp dụng với nguyên nhân lỗi thứ ba. 

Đối với ngôn ngữ đơn lập như tiếng Việt, vấn đề mới phát sinh là không
thể xác định rõ ràng ranh giới từ. Với tiếng Anh và các ngôn ngữ biến
cách, khoảng trắng được dùng để phân cách hai từ. Trong tiếng Việt,
khoảng trắng được dùng để phân cách hai tiếng. Ngoài ra, việc định
nghĩa từ trong tiếng Việt vẫn chưa thống nhất. LV này sử dụng ``từ''
như là ``từ từ điển''.

Để bắt lỗi chính tả trong tiếng Việt, có thể dựa trên nhiều cách khác
nhau. Cách đầu tiên đơn giản là tìm ranh giới từ cho tiếng Việt, sau
đó chuyển về bài toán bắt lỗi chính tả của tiếng Anh. Một cách khác
là bắt lỗi mà không cần tách từ. XXX
Cách khác dựa trên ý tưởng ``phân tích cú pháp câu, nếu ta không thể
phân tích cú pháp của câu, nghĩa là câu đó sai chính tả''.

LV này làm theo hướng tách từ, sau đó xác định lỗi chính tả.

\section{Previous works}

Điều hiển nhiên dễ thấy là nếu câu bị sai chính tả thì ta không thể
tách từ đúng được. Vấn đề ở đây là phải tách từ trong câu bị sai chính
tả.

Oflazer khi xử lý tách tiếp vĩ ngữ trong tiếng Thổ Nhĩ Kỳ gặp trường
hợp khá giống với trường hợp này. Tác giả phải tách các tiếp vĩ ngữ
tiếng Thổ Nhĩ Kỳ trong điều kiện từ đó bị sai chính tả. Do đặc tính
ngôn ngữ chấp dính (agglunative) nên khó có thể phân biệt đâu là tiếp
vĩ ngữ, cũng như không thể phân biệt đâu là từ trong một chuỗi tiếng
trong tiếng Việt. Tác giả dùng một hàm độ đo, tạo ra các tiếp đầu ngữ
có khả năng thay thế dựa trên độ đo này, sau đó sử dụng WFST để tìm
chuỗi tiếp vĩ ngữ thích hợp nhất.

Nhận dạng tiếng nói tiếng Anh cũng gặp trường hợp tương tự. Sau công
đoạn xử lý âm thanh, ta nhận dạng được một chuỗi các âm tiết
(phoneme). Phải làm cách nào đó để nhóm các âm tiết này thành từ. Do
âm thanh thường bị nhiễu, nên các âm tiết có thể không chính xác hoàn
toàn.

Dựa trên hai cái này, có thể thấy giải pháp cho việc tách từ khi bị
sai chính tả, là phát sinh một loạt các từ có khả năng thay thế, với
hy vọng trong tập từ này sẽ có từ đúng chính tả, thay thế từ sai chính
tả ban đầu. Sau đó sử dụng tách từ tìm một cách tách tốt nhất. Sau khi
tìm được cách tách từ, ta có thể tra từ điển để tìm xem từ nào bi sai.


\subsection{Tách từ}

Bài toán tách từ cho ngôn ngữ đơn lập đã được đặt ra từ lâu, chủ yếu
để giải quyết cho tiếng Hoa, tiếng Nhật. 

Chao-Huang Chang sử dụng WFST để tách từ. Training bằng EM dựa trên
segmentation đúng nhất.

Mosur K. Ravishankar đề nghị tạo ra ``word lattice'' sau đó sử dụng
thuật toán tìm đường đi ngắn nhất để giải quyết.

Le An Ha sử dụng ngram để tách từ.

Chuynu Kit kết hợp ngram, lập trình quy hoạch động để tách từ. Xài
soft-count thay vì ``hard-count'' như Chang. Chunyu còn đề nghị dùng
case-based learning.


\subsection{Training tách từ}





\chapter{Cài đặt}


\chapter{Kết luận}
\label{cha:conclusion}



\end{document}
