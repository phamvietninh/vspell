\documentclass[a4paper]{book} % -*- mode: latex auto-fill:t -*-

%\usepackage[hmargin={2cm,2cm}]{geometry}
\usepackage{graphicx}
\usepackage{amsmath}
\renewcommand{\rmdefault}{utm}
\renewcommand{\sfdefault}{uhv}
\renewcommand{\ttdefault}{ucr}
\usepackage[utf8]{vietnam}

\newtheorem{algo}{Thuật toán}

\title{Luận văn tốt nghiệp\\Đề tài\\Nhảm nhí}
\author{Nguyễn Thái Ngọc Duy\\0012020}

\begin{document}
%\maketitle{}
%\tableofcontents{}
%\listoffigures{}

\chapter{Giới thiệu}
\label{cha:intro}
%% \begin{center}
%% I kept the right ones out

%% And let the wrong ones in

%% Had an angel of mercy

%% To see me through all my sins

%% There were times in my life

%% When I was goin' insane

%% Tryin' to walk through the pain

%% ****

%% And when I lost my grip

%% And I hit the floor

%% Yeah, I thought I could leave

%% But couldn't get out the door

%% I was so sick n' tired

%% Of livin' a lie

%% I was wishing that I would die

%% ****

%% It's amazing

%% With the blink of an eye

%% You finally see the light

%% It's amazing

%% That when the moment arrives

%% You know you'll be alright

%% It's amazing

%% And I'm saying a prayer

%% To the desperate hearts tonight


%% ****


%% That one last shot's a Permanent Vacation

%% And a how high can you fly with broken wings

%% Life's a journey - not a destination

%% And I just can't tell just what tomorrow brings


%% ****


%% You have to learn to crawl

%% Before you learn to walk

%% But I just couldn't listen

%% To all that righteous talk

%% I was out on the street

%% Just tryin' to survive

%% Scratchin' to stay alive


%% ****


%% "To all of you people out there

%% Wherever you are - remember:

%% The light at the end of the tunnel

%% May be you - goodnight" 
%% \end{center}

\chapter{Thông tin nền}
\label{cha:background}

\section{Tổng quan về lỗi chính tả}

Văn bản nhập vào thường hay bị lỗi. Các nguyên nhân gây ra lỗi là:
Theo \cite{Chang} thì lỗi là:
\begin{itemize}
\item Giống phiên âm
\item Giống hình dạng
\item Cách gõ tương tự nhau
\item Giống nghĩa
\end{itemize}

Luận văn này chỉ giải quyết ba lỗi đầu.

\section{Xử lý lỗi chính tả}

Có hai hướng chính:
\begin{itemize}
\item Tìm lỗi, đề nghị cách sửa chữa.
\item Tự động sửa lỗi.
\end{itemize}

Tìm lỗi dựa chủ yếu trên từ điển. So từng từ với từ điển, những từ
không có trong từ điển là những từ có khả năng bị lỗi. Sau đó dựa trên
một số heuristic hoặc các độ đo để tìm ra từ gần đúng với từ đó, làm
từ đề nghị.

Tự động sửa lỗi chủ yếu dựa trên một tập các từ hay bị lỗi (then-than,
there-their \ldots). Sử dụng ngữ cảnh xung quanh để xác định từ đúng
hay sai.

Đối với ngôn ngữ đơn lập như tiếng Việt, vấn đề mới phát sinh là không
thể xác định rõ ràng ranh giới từ. Với tiếng Anh và các ngôn ngữ biến
cách, khoảng trắng được dùng để phân cách hai từ. Trong tiếng Việt,
khoảng trắng được dùng để phân cách hai tiếng. Ngoài ra, việc định
nghĩa từ trong tiếng Việt vẫn chưa thống nhất. Trong luận văn này sử
dụng ``từ'' như là ``từ từ điển''.

Để bắt lỗi chính tả trong tiếng Việt, có thể dựa trên nhiều cách khác
nhau. Cách đầu tiên đơn giản là tìm ranh giới từ cho tiếng Việt, sau
đó chuyển về bài toán bắt lỗi chính tả của tiếng Anh. Cách khác
là bắt lỗi mà không cần tách từ. XXX
Cách nữa dựa trên ý tưởng ``phân tích cú pháp câu, nếu ta không thể
phân tích cú pháp của câu, nghĩa là câu đó sai chính tả'' \cite{iccc}.

Luận văn này làm theo hướng tách từ, sau đó xác định lỗi chính tả.

\section{Những nghiên cứu trước đây}

Điều hiển nhiên dễ thấy là nếu câu bị sai chính tả thì ta không thể
tách từ đúng được. Vấn đề ở đây là phải tách từ trong câu bị sai chính
tả.

\cite{Oflazer} khi xử lý hình thái trong tiếng Thổ Nhĩ Kỳ gặp trường
hợp khá giống với trường hợp này. Tác giả phải tách hình thái từ
trong điều kiện từ đó bị sai chính tả. Do đặc tính
ngôn ngữ chắp dính\footnote{agglunative language}, số tiếp vĩ ngữ
nhiều, liên tiếp nhau, gây khó khăn cho việc nhận dạng tiếp vĩ ngữ,
cũng như không thể phân biệt những tiếng nào hợp thành một từ trong
một chuỗi tiếng trong tiếng Việt. Tác giả dùng một hàm độ đo, tạo ra
các tiếp đầu ngữ có khả năng thay thế dựa trên độ đo này, sau đó sử
dụng WFST để tìm chuỗi tiếp vĩ ngữ thích hợp nhất.

Nhận dạng tiếng nói tiếng Anh cũng gặp trường hợp tương tự
\cite{Ravishankar}. Sau công 
đoạn xử lý âm thanh, ta nhận dạng được một chuỗi các âm tiết
(phoneme). Phải làm cách nào đó để nhóm các âm tiết này thành từ. Do
âm thanh thường bị nhiễu, nên các âm tiết có thể không chính xác hoàn
toàn.

\cite{Chang} Có thể dùng nhiều LM khác nhau như character
bigram, word bigram, inter-word character bigram (IWCB), POS bigram,
word class bigram. Các bước thực hiện:
\begin{enumerate}
\item Input and Clause Segmentation. Dùng cách ký tự như `?' `!' `,'
  `:' `.' \ldots{} để tách.
\item Composite Confusing Character Stubstitution. 
\item Language Model Evaluation. Xài Modified-IWCB và SA-class
  bigram. IWCB là, những từ cùng ký tự đầu tiên được nhóm thành 1
  class, những từ cùng ký tự cuối cùng được nhóm thành class. Bigram
  dựa trên hai nhóm này. ``Modified'' là tính luôn word freq. SA-class
  bigram dùng SA algorithm để nhóm.
\item Comparision and Correction.
\item Output.
\end{enumerate}


Dựa trên hai cái này, có thể thấy giải pháp cho việc tách từ khi bị
sai chính tả, là phát sinh một loạt các từ có khả năng thay thế, với
hy vọng trong tập từ này sẽ có từ đúng chính tả, thay thế từ sai chính
tả ban đầu. Sau đó sử dụng tách từ tìm một cách tách tốt nhất. Sau khi
tìm được cách tách từ, ta có thể tra từ điển để tìm xem từ nào bị sai.


\subsection{Tách từ}

Bài toán tách từ cho ngôn ngữ đơn lập đã được đặt ra từ lâu, chủ yếu
để giải quyết cho tiếng Hoa, tiếng Nhật. 

\cite{Sproat} sử dụng WFST để tách từ, huấn luyện bằng EM dựa trên
cách tách từ đúng nhất.

\cite{wordseg} sử dụng giải pháp tương tự như \cite{Sproat} nhưng cải
tiến bằng cách cáp dụng Neural Networks để giải quyết nhập nhằng dựa
vào POS. Nghiên cứu đề cập đến một số vấn đề tiền xử lý tiếng Việt như
xác định tên riêng, từ láy, phân tích hình thái.

\cite{Ravishankar} đề nghị tạo ra lưới từ\footnote{word lattice} sau
đó sử dụng thuật toán tìm đường đi ngắn nhất để tìm cách tách từ tốt
nhất dựa trên 2-gram hoặc 3-gram.

\cite{LAH} sử dụng n-gram và lập trình quy hoạch động để tách từ,
không dùng từ điển. Giải pháp này tương tự như \cite{softcount}, tuy
nhiên chỉ tính xác suất cách tách từ tốt nhất thay vì tính tổng xác
suất mọi cách tách từ như trong \cite{softcount}.


\cite{Chunyu} kết hợp ngram, lập trình quy hoạch động để tách từ. Xài
soft-count thay vì ``hard-count'' như \cite{Chang}. \cite{Chunyu} còn
đề nghị dùng case-based learning.

\subsection{Huấn luyện tách từ}

Có thể huấn luyện dựa trên dữ liệu mẫu, hoặc dữ liệu thô. Do hầu hết
các phương pháp tách từ đều dựa trên ngram ($n\ge 1$) nên rất cần có
khối lượng dữ liệu huấn luyện lớn nhằm bao quát hết các gram. Dữ liệu
mẫu thường không đủ để huấn luyện. Giải pháp chủ yếu là huấn luyện dựa
trên dữ liệu thô.

Thuật toán thường dùng nhất để huấn luyện trên dữ liệu thô là thuật
toán EM. Nhiều người đã cố gắng cải tiến EM theo nhiều cách khác nhau
nhằm nâng cao chất lượng huấn luyện, đồng thời hạn chế những khuyết
điểm của EM. 

\cite{self-supervised} đề nghị cách giải quyết hạn chế ``tối ưu cục
bộ'' của EM, bằng cách phân phối lại từ vựng sau mỗi lần chạy và khởi
động lại EM với điểm khởi đầu tốt hơn.

\cite{text-tiling} sử dụng một lượng nhỏ ngữ liệu huấn luyện có chất
lượng cao (\textit{seed set}) và một lượng lớn ngữ liệu có chất lượng
không thật bảo đảm (\textit{training set}), xử lý qua 4 bước:
\begin{enumerate}
\item \textbf{Phân đoạn training set}
\item \textbf{Phân hạng training set}
\item \textbf{Tổ hợp training set}
\item \textbf{Language model pruning}
\end{enumerate}

\chapter{Cài đặt}

Chương trình gồm hai phần: phần bắt lỗi chính tả và phần huấn luyện.

\section{Trình bắt lỗi chính tả}


\subsection{Quy trình chung}
\label{sec:spellcheck}

Việc bắt lỗi chính tả của một văn bản được xử lý lần lượt qua những
bước sau:
\begin{enumerate}
\item \textbf{Tiền xử lý} Tách văn bản thành những đoạn ngắn. Tách
  đoạn thành từng tiếng. Đánh dấu các ký hiệu, dấu ngắt dòng, các số,
  tên riêng \ldots
\item \textbf{Bắt lỗi tiếng} Kiểm tra các tiếng với các tiếng đã có
  trong từ điển. Báo lỗi những tiếng không có trong từ điển. Sau đó 
  \textbf{Đưa ra giải pháp thay thế}.
\item \textbf{Tạo lưới từ} Tìm ra mọi từ có thể có trong câu. Xem
  giải thích về lưới từ bên dưới. Lồng trong phần này là phần
  \textbf{Phát sinh từ thay thế}
\item \textbf{Tách từ} Dựa vào lưới từ, đưa ra cách tách từ tốt nhất.
\item \textbf{Bắt lỗi từ} Dựa vào từ điển và cách tách từ đã có,
  tìm những từ nào không có trong từ điển. Những từ này được xem là từ
  sai. Sau đó \textbf{Đưa ra giải pháp thay thế}
\end{enumerate}

\subsection{Tiền xử lý}
\label{sub:preprocess}

Dựa vào flex để tách thành các token. Mục đích là để tách các dấu câu
ra khỏi tiếng. Ví dụ \verb#"chào"# sẽ được tách thành 3 token
\verb#"#, \verb#chào# và \verb#"#. Quy tắc được dùng như sau:

FIXME.

\subsubsection{Tách câu}

Dựa vào các dấu câu để ngắt câu ra thành từng đoạn để xử lý. Mỗi
đoạn sẽ được xử lý độc lập với nhau. Đoạn ở đây có thể là một câu,
nhưng cũng có thể là một 
phần của câu. Luận văn này sẽ dùng từ ``câu'' để ám chỉ ``đoạn''. Nếu
các thông tin ở mức cao hơn được sử dụng (như thông tin cú pháp, ngữ
nghĩa \ldots) thì phải thật sự xử lý trên câu chứ không phải trên
đoạn. Các đoạn được phân cách bởi các dấu câu (\verb#.# \verb#,# \verb#;# \verb#(# \verb#)# \ldots)

\subsubsection{Chuẩn hóa NOTIMPL}

Do trong tiếng Việt có nhiều từ có thể viết theo các
cách khác nhau về vị trí dấu thanh điệu, ví dụ, ``hoà'' và
``hòa''. Bởi vậy cần phải chuẩn hóa sao cho chương trình xem ``hòa''
và ``hoà'' là một. Giải pháp đưa ra là tách dấu thanh điệu ra khỏi
từ, biểu diễn dấu thanh điệu bằng ký tự đầu tiên trong từ. Như vậy,
``hoà'' và ``hòa'' đều có cùng cách biểu diễn là ``2hoa'', trong khi
đó ``hoa'' được biểu diễn là ``0hoa'', ``hồng'' được biểu diễn là
``2hông''.

\subsubsection{Tên riêng NOTIMPL}

Xử lý chữ viết hoa. Chữ viết hoa dùng để biểu diễn tên riêng, hoặc
dùng cho chữ đứng đầu câu. Do đó cần phân biệt chữ đầu câu có phải là
chữ bắt đầu tên riêng hay không. Ngoài ra, cần xác định tên riêng khi
tìm được chữ viết hoa bắt đầu tên riêng. Các văn bản tiếng Việt chưa
hoàn toàn thống nhất về quy tắc viết hoa. Ví dụ, có tài liệu dùng
``Cộng hoà Xã hội Chủ nghĩa Việt Nam'', nhưng có tài liệu lại dùng
``Cộng Hoà Xã Hội Chủ Nghĩa Việt Nam''. HELPME.
Có thể dùng heuristic như sau: Nếu bắt đầu bằng chữ viết hoa, và chữ
đầu được viết thường cùng những từ kế tiếp của nó là một từ trong từ
điển, thì xem như đó là một từ. Nếu một chuỗi từ viết hoa liên tục
(hai từ trở lên) thì xem như đó là một từ (một tên riêng), và được
đánh dấu từ lạ ``Unknown word''.

Cách tiếp cận này không hoàn hảo. Lỗi sai của heuristic này
có thể ảnh hưởng nặng nề đến chất lượng tách từ và bắt lỗi. Nên đánh
dấu theo kiểu ``80\verb#%# UNK, 10\verb#%# là tổ hợp từ \ldots''. Cần
nghiên cứu sâu hơn về vấn đề này.

Cần tham khảo thêm các tài liệu về vấn đề này.

\subsubsection{Từ láy NOTIMPL}

Hổng thèm làm :-) Vấn đề từ láy cũng tương tự như  tên riêng. Giải
pháp có thể áp dụng như của LV anh Toàn. Nhưng những ``từ láy''
``nhưng những'' có thể giết chết module tách từ một cách êm dịu. Chắc
nên áp dụng giải pháp ``80 từ láy, 20 từ thường''. Không biết được hay
không, phiêu quá \ldots

\subsubsection{Từ nước ngoài, các ký hiệu, \ldots NOTIMPL}

Coi như từ sai chính tả.
Xử lý tiếng nước ngoài, các ký hiệu chuyên ngành, các từ viết tắt. Do
trình bắt lỗi không có kiến thức về các lĩnh vực chuyên 
ngành, cũng như các thứ tiếng trên thế giới, nên việc áp dụng tri thức
để phân loại là điều hết sức khó khăn. Giải pháp được dùng ở đây là áp
dụng hàm độ đo $ed$ trên từ đang xét và từ điển. Nếu từ không có trong
từ điển và giá trị hàm $ed$ lớn hơn 2. Từ đó được xem như là từ lạ
``Unknown word''. Các con số được đánh dấu riêng bằng mã ``Cardinal word''.
``Số'' ở đây là bất cứ chữ nào bắt đầu bằng số. Ví dụ, ``0lit'',
``0.2'', ``0-4'' \ldots{} đều được coi là số. Với những từ có $ed$ nhỏ
hơn 2, từ đó được xem là sai chính tả. 

\subsection{Phát sinh từ thay thế}

Phần này được sử dụng trong lúc tạo lưới từ và kiểm tra chính
tả. Mục đích là, cho trước một từ, phát sinh những từ ``gần giống''
với từ đó. Việc định nghĩa như thế nào là ``giống'' ở đây dựa theo các
nguyên nhân gây ra lỗi chính tả:
\begin{itemize}
\item Lỗi phát âm. Ví dụ, ``vu'' và ``du''. Lỗi phát âm phụ thuộc vào
  cách phát âm của từng vùng. CITEZ liệt kê các trường hợp lỗi thông
  dụng nhất. Những quy tắc này được áp dụng để, ví dụ, từ ``du'' ta
  tạo ra ``vu''.
\item Do lỗi bàn phím. Có thể do gõ nhầm những phím lân cận. Ví dụ: gõ
  ``tôi'' thành ``tôu''. Khắc phục lỗi này dựa vào bố trí phím trên
  bàn phím. Độ đo là khoảng cách từ phím tạo ra ký tự trong từ cho
  trước và những từ chung quanh. Thông thường lỗi này chỉ xảy ra một
  lần trong mỗi từ.
\item Lỗi OCR. XXXX. ko thèm xử lý, kêu OCR tự xử đi :-D
\item Lỗi sai về mặt ngữ nghĩa, cú pháp. Lỗi này không xử lý.
\item Lỗi không rõ nguyên nhân. Với dạng lỗi này, ta dùng hàm độ đo,
  tính số lần thêm/xóa/thay đổi/hoán vị mỗi ký tự giữa hai từ. Hàm độ đo
  được dùng được nêu trong \cite{Oflazer}, sẽ được trình bày lại bên
  dưới.
\end{itemize}


Hmm.. có lẽ nên gắng thêm vào các từ phát sinh một ``hệ số chính
xác''. Từ gốc là từ có hệ số chính xác cao hơn so với các từ được phát
sinh. Good or bad? Hmmm


\subsubsection{Lỗi phát âm}



\subsubsection{Lỗi bàn phím NOTIMPL}

Sơ đồ bố trí của bàn phím EN-US được dùng. Do thông thường chỉ gặp một
lỗi này mỗi từ, nên chương trình chỉ lưu danh sách những phím lân cận
với từng phím, dựa trên bàn phím EN-US. Ví dụ: \texttt{A} $\rightarrow$
(\texttt{S},\texttt{Q},\texttt{W},\texttt{X},\texttt{Z}).

\subsubsection{Lỗi không rõ nguyên nhân NOTIMPL}

\label{algo:ed}
Lỗi này dựa vào {\em độ đo khoảng cách hiệu chỉnh} được đề cập
trong \cite{Oflazer}. Các thao tác hiệu chỉnh được đo gồm {\em chèn,
xóa, thay thế một ký tự} hoặc {\em hoán vị hai ký tự kề nhau}, để có
thể chuyển đổi từ này thành từ kia. Đặt $X = x_1, x_2, \ldots, x_m$ và
$Y = y_1,y_2,\ldots,y_n$ là hai chuỗi có độ dài tương ứng là $m$ và
$n$. $X[i]\quad(Y[j])$ biểu diễn chuỗi con ban đầu của X (Y) từ đầu từ
đến ký tự thứ $j$. Cho $X$ và $Y$, độ đo $ed(X[m],Y[n])$ được tính như
sau:
\begin{equation}
\begin{array}{rll}
  ed(X[i+1],Y[j+1]) &= ed(X[i],Y[j]) & \text{nếu $x_{i+1}=y_{j+1}$
  (ký tự cuối như nhau)}\\
                    &= 1+min\{ed(X[i-1],Y[j-1]), & \text{nếu $x_i=y_{j+1}$}\\
                    &\qquad\qquad ed(X[i+1],Y[j]), & \text{và $x_{i+1}=y_j$}\\
                    &\qquad\qquad ed(X[i],Y[j+1])\}\\
                    &= 1+min\{ed(X[i],Y[j]),&\text{trường hợp khác}\\
                    &\qquad\qquad ed(X[i+1],Y[j]),\\
                    &\qquad\qquad ed(X[i],Y[j+1])\}\\
  ed(X[0],Y[j])     &=j & 0 \le j \le n\\
  ed(X[i],Y[j])     &=i & 0 \le i \le n\\\\
  ed(X[-1],Y[j])    &=ed(X[i],Y[-1]) = max(m,n)&\text{Biên}
\end{array}
\end{equation}

Thuật toán có thể được khử đệ quy khi cài đặt bằng quy hoạch động.

XXXX. cho ví dụ, ghi thuật toán.
\begin{algo} Tính hàm ed()
  
\end{algo}

Ví dụ:

\subsection{Tạo lưới từ}
\label{sub:lattice}
Lưới từ\footnote{word lattice} là một đồ thị có hướng không chu trình,
với các nút là các từ trong câu, cạnh là đường nối giữa hai từ kề
nhau, hướng thể hiện hướng của câu (từ trái sang phải). Lưới từ
chứa tất cả 
các từ có khả năng xuất hiện trong câu. Các từ được liên kết với nhau
theo trật tự trong câu. Khi duyệt từ nút gốc đến nút đích, ta sẽ được
một cách tách từ cho câu.

\begin{figure}[htbp]
  \centering
  \includegraphics[width=\textwidth]{wordlattice}
  \caption{Lưới từ của câu ``Học sinh học sinh học''}
  \label{fig:wordlattice}
\end{figure}

Khi tạo lưới từ trong chương trình bắt lỗi chính tả, thuật toán không
chỉ phát sinh những từ được tạo từ đoạn, mà còn những từ {\em có thể
có} được phát sinh từ đoạn.

\begin{figure}[htbp]
  \centering
  \includegraphics[width=\textwidth]{wordlattice1}
  \caption{Lưới từ mở rộng của câu ``Học sinh học sinh học''}
  \label{fig:wordlattice1}
\end{figure}

Lưới từ được tạo bằng thuật toán Viterbi. Mỗi tiếng trong câu được
duyệt qua để tìm ra tất cả các từ có thể có trong đoạn. Sau đó tập hợp
các từ này lại. 

XXXX. thuật toán tạo lưới từ.
\begin{algo} Tạo lưới từ

Cho câu S có n tiếng. Duyệt i từ 1 đến n:
\begin{itemize}
\item Tạo state gốc cho nút i.
\item Xét các state, nếu tiến thêm được một bước thì lưu lại state mới.
\item Nếu không tiến được thì xóa state.
\item Nếu hoàn tất một từ thì lưu lại.
\end{itemize}
  
\end{algo}

Ngoài lưới từ, ta có thể tạo lưới 2-từ từ lưới từ. Lưới 2-từ tương tự
như lưới từ, tuy nhiên mỗi nút là một cặp 2 từ đi liền nhau trong câu. Thuật toán
tạo lưới 2-từ được nêu trong \cite{Ravishankar}, được tóm tắt lại như sau:

\begin{figure}[htbp]
  \centering
  \includegraphics[width=\textwidth]{wordlattice2}
  \caption{Lưới 2-từ của câu ``Học sinh học sinh học''}
  \label{fig:wordlattice2}
\end{figure}

\begin{algo}Tạo lưới n-từ từ lưới (n-1)-từ

\begin{enumerate}
\item Nếu nút ($w$) có $n$ từ đứng liền trước nó ($w_i$),
  $i=1,2,\ldots,n$ trong lưới từ gốc, nó sẽ được lặp lại $n$ lần trong
  lưới từ mới, tên là ($w_{i}w$), tương ứng với $i=1,2,\ldots,n$.
\item Nếu ($w_i$) nối với ($w_j$) trong lưới từ gốc, nối tất cả
  ($w_xw_i$) với ($w_iw_j$) $x$ bất kỳ.
\item Giá trị của ($w_iw_j$) là giá trị của cạnh ($w_i$) ($w_j$) trong
  lưới từ cũ.
\item Giá trị của cạnh ($w_iw_j$) ($w_jw_k$) là 3-gram của $w_i$, $w_j$
  và $w_k$.
\end{enumerate}
\end{algo}

Các lưới n-từ có đặc điểm là tăng nhanh số nút và số cạnh, nhưng số
tầng vẫn không đổi (Đồ thị càng ngày càng ``mập'', nhưng ``cao''
không nổi ;-) )


\subsubsection{Lưu lưới từ}

Các nút trong lưới từ được lưu tập trung vào một mảng
(WordEntries). Truy cập nút được thực hiện thông qua index của mảng
này. Mỗi nút chứa các thông tin về từ của nút đó, bao gồm vị trí từ,
số tiếng, index trong WordEntries\ldots{}

Words là một mảng tương ứng với vị trí từng tiếng trong câu. Vị trí
của mỗi tiếng sẽ được liên kết với một danh sách các từ bắt đầu tại
tiếng đó. Ta có thể coi danh sách này như danh sách các cạnh nối từ
nút đang xét đến các nút kế tiếp.


\subsection{Tách từ}
\label{sub:wordseg}

Dùng thuật toán tìm kiếm theo độ ưu tiên\footnote{Priority-First
Search---PFS} để tìm đường đi ngắn nhất trên đồ thị. Khoảng cách giữa hai
điểm trong đồ thị là xác suất 2-gram của hai từ. Bài toán tìm đường
đi ngắn nhất, có thể áp dụng PFS, Prime, Djisktra. PFS được chọn vì
lưới từ, qua khảo sát, có thể được coi là một đồ thị thưa.

Để áp dụng 3-gram thay vì 2-gram, ta sẽ sử dụng lưới 2-từ. Sau đó áp
dụng thuật toán PFS cho đồ thị mới.

Cách làm này không thể thực hiện với n-gram ($n > 3$) vì số nút/cạnh
trong đồ thị sẽ tăng đáng kể. Trong trường hợp đó ta nên sử dụng
thuật toán A*

XXXX. Nếu WFST cộng viterbi, beam pruning \ldots thì sai nhỉ?


\subsection{Tìm từ thay thế}

Đối với tiếng Anh, bước này dùng thuật toán như doublephone, Soundex
\cite{soundex}, Phonetex \cite{phonetex}. Ý tưởng 
là dựa vào heuristic, thay thế các nhóm ký tự trong từ thành một ký
hiệu tương trưng cho một âm. Sau đó so sánh độ hiệu chỉnh giữa âm của
từ sai và các âm của từ trong từ điển. Do tiếng Việt đọc sao ghi vậy,
nên bước chuyển từ từ sang âm có thể bỏ qua. Bước còn lại là áp dụng
thuật toán đo độ hiệu chỉnh giữa hai từ như đã nêu trong \ref{algo:ed}
Ta có thể so sánh từng ký tự, hoặc so sánh theo âm (ví dụ, ``th'' thay
vì ``t'' và ``h'').

Chương trình sẽ liệt kê XXXX (chọn theo tiêu chí nào?, 10 từ lớn nhất
hay sao đây)

XXXX. thuật toán tính ed.


\section{Trình huấn luyện}

\subsection{Tiền xử lý}
Giống như phần \ref{sub:preprocess}


\subsection{Tạo Lưới từ}
Giống như phần \ref{sub:lattice}, nhưng chỉ xét những từ nào thực sự
có trong câu.


\subsection{Thống kê n-gram}
\label{sub:wordcount}

Nếu dùng PFS để tách từ, ta chỉ có 1 cách tách từ tốt nhất. Việc đếm
từ sẽ dựa trên cách tách từ này.

Như đã nói, có thể dùng WFST hoặc dùng thuật toán trong
\cite{softcount} (tạm gọi là thuật toán ``soft-count'') để tách
từ. WFST phải dùng kèm với beam pruning để tránh bùng nổ số tổ hợp các
cách tách từ. Sau khi dùng WFST, ta còn n cách tách từ, có thể đếm
fractional count (được đề cập bên dưới) trên các cách tách từ này.

Trường hợp nhiều cách tách từ, ta có thểm đếm từ trên tất cả các cách
tách, thay vì chỉ đếm trên cách tách từ tốt nhất. Cách này phản ánh
tầm ảnh hưởng của các từ tốt hơn. Ví dụ, ta có 3 cách tách từ với xác
suất các cách tách từ tương ứng lần lượt là 0.5, 0.4 và 0.1. Cách chỉ
dùng cách tách từ tốt nhất sẽ chỉ tính những từ trong cách tách từ
đầu, với giá trị mỗi từ là 1.0. Cách tính này ``dồn phiếu'' của 2 cách
sau cho cách đầu. Trường hợp sau, các từ trong cách 2 và 3 vẫn được
tính. Số đếm của mỗi từ không còn là 1, mà là xác suất của cách tách
từ chứa từ đó. Trở lại ví dụ, các từ trong cách một sẽ được cộng thêm
0.5 thay vì 1. Ngoài ra các từ trong cách 2 và 3 lần lượt được cộng
0.4 và 0.1. Dễ thấy, các từ trong cách tách từ thấp sẽ không tăng số
đếm đáng kể, do đó không thể gây ảnh hưởng lớn đến quyết định tách từ.
Cách cộng dồn số như vậy được gọi là ``fractional count'' (hay trong
\cite{softcount} gọi là ``soft-count'').

Luận văn này thuật toán soft-count để đếm mọi cách tách từ. Soft-count
thực tế không kết xuất ra một cách tách từ cụ thể nào (và do vậy nên
cũng không thể áp dụng để tìm cách tách từ tốt nhất được!). Thay vào
đó, thuật toán đếm mọi từ thể có. Thuật toán được mô tả trong
\cite{softcount} không dùng từ điển. Mọi chuỗi con trong câu đều được
cho là từ. Điểm này, vừa là điểm mạnh vì không cần dùng từ
điển, nhưng cũng là điểm yếu vì yếu tố này làm giảm độ chính xác một
cách đáng kể (khoảng 20\verb#%#, như kết quả trong \cite{softcount}).

Giả sử câu S có $n$ tiếng, $2^{n-1}$ cách tách từ khác nhau, xác suất
mỗi cách tách từ là $p(1),p(2),\ldots,p(n)$.
Với mỗi từ trong một cách tách từ, ta cộng thêm một khoảng
$\displaystyle\frac{p(i)}{\sum_{i=1}^n{p(i)}}$ cho từ đó. Soft-count dùng
lập trình quy hoạch động để thực hiện quá trình này.

Thuật toán được dùng ở đây là thuật toán soft-count, được hiệu chỉnh
sử dụng từ điển và lưới từ để hạn chế những từ không phải là từ.

Cho lưới từ của câu S. Gọi $L(W)$ là tập những từ nối đến nút
$W$. Tương tự, $R(W)$ là tập những từ được nối đến từ nút $W$.
Với mỗi nút $W$ trong S, tổng xác suất các cách tách từ có chứa nút
$W$ là:
$$P(W)=P^{left}(W)p(W)P^{right}(W)$$
Trong đó:
\begin{itemize}
\item $P^{left}(W)$ là tổng xác suất các cách tách từ tính từ đầu câu
  đến $W$.
\item $P^{right}(W)$ là tổng xác suất các cách tách từ tính từ $W$ đến
  hết câu.
\end{itemize}

$$
P^{left}(W) = \left\{
    \begin{array}{ll}
      p(W)&\text{nếu W là nút head}\\
      \displaystyle\sum_{W' \in L(W)}p(W')P^{left}(W')&\text{ngược lại}\\
    \end{array}
  \right.
$$

Tương tự

$$
P^{right}(W) = \left\{
    \begin{array}{ll}
      p(W)&\text{nếu W là nút tail}\\
      \displaystyle\sum_{W' \in R(W)}p(W')P^{right}(W')&\text{ngược lại}\\
    \end{array}
  \right.
$$

\begin{algo}Tính $P^{left}$

\begin{enumerate}
\item Đặt $P^{left}(head) = p(head)$
\item Đặt $P^{left}(W) = 0$ với mọi $W$ còn lại.
\item Duyệt lần lượt các nút $W$ theo thứ tự từ trái sang phải, tính
  theo vị trí bắt đầu của $W$ trong câu. Với $W' \in R(W)$
  cộng thêm $p(W)P^{left}(W)$ vào $P^{left}(W')$
\end{enumerate}
\end{algo}

Thuật toán tương tự được áp dụng để tính $P^{right}$.

Sau khi tính được $P^{left}$ và $P^{right}$, ta có thể tính fractional
count cho các từ trong câu bằng cách duyệt tất cả các nút trong lưới từ,
cộng thêm vào $\displaystyle\frac{P(W)}{P^{left}(tail)}$ cho từ
$C$. Thực tế, ta sẽ lồng bước này vào trong thuật toán tính
$P^{right}$, vì thuật toán cũng phải duyệt qua tất cả các từ.

\begin{algo}Tính $P^{right}$
  
\begin{enumerate}
\item Đặt $P^{right}(tail) = p(tail)$
\item Đặt $P^{right}(W) = 0$ với mọi nút còn lại.
\item Duyệt tất cả các nút $W$ từ phải sang trái, tính theo vị trí kết
  thúc của $W$ trong câu.
  \begin{enumerate}
  \item Với $W' \in L(W)$, cộng thêm $p(W)P^{right}(W)$ vào
    $P^{right}(W')$
  \item Tính fractional count cho $W$ theo công thức
    $\displaystyle\frac{P^{left}(W)p(W)P^{right}(W)}{P^{left}(tail)}$
  \end{enumerate}
\end{enumerate}
\end{algo}


Tuy nhiên, thuật toán trên (cũng như thuật toán gốc) sử dụng
uni-gram, trong khi trình bắt lỗi lại dùng 2-gram. Để cho phép thuật
toán dùng 2-gram, ta có thể tạo một lưới 2-từ như cách của
\cite{Ravishankar} khi muốn tách từ dựa trên 3-gram. Tuy nhiên, để áp
dụng cách này với 3-gram đòi hỏi phải tạo lưới 3-từ! Số lượng nút
trong lưới 3-từ nhiều hơn nhiều so với lưới từ gốc, làm giảm tính hiệu
quả của thuật toán. 

Thay vì vậy, thuật toán được hiệu chỉnh để áp dụng 2-gram với lưới từ
thông thường. Thay vì dùng giá trị nút để tính, ta dùng giá trị cạnh
để tính. $p(W)$ sẽ được thay bằng $p(W/W')$.

$$P(W)=P^{left}(W)P^{right}(W)$$

$$
P^{left}(W) = \left\{
    \begin{array}{ll}
      p(W)&\text{nếu W là nút head}\\
      \displaystyle\sum_{W' \in L(W)}p(W/W')P^{left}(W')&\text{ngược lại}\\
    \end{array}
  \right.
$$

Tương tự

$$
P^{right}(W) = \left\{
    \begin{array}{ll}
      p(W)&\text{nếu W là nút tail}\\
      \displaystyle\sum_{W' \in R(W)}p(W'/W)P^{right}(W')&\text{ngược lại}\\
    \end{array}
  \right.
$$

$P(W)$ đại diện cho xác suất tất cả các cách tách từ đi qua nút
$W$. Do ta huấn luyện 2-gram, nên cần xác suất $P(W/W')$ chứ không cần
$P(W)$. Nếu $W$ và $W_{+1}$ chỉ có một cạnh, $P(W)$ cũng chính là
$P(W/W_{+1})$. Nếu có nhiều hơn một cạnh, ta cần dùng cách khác để
tính $P(W/W')$. Với thuật toán tính $P^{right}$ hiện có, ta cộng dồn
$p(W/W')P^{right}$ vào cho $P^{right}(W)$. Vậy trong quá trình cộng
dồn ta có thể tính ngay $P(W/W')$ bằng cách nhân giá trị cộng dồn với
$P^{left}(W)$. Thuật toán tính $P^{left}$ và $P^{right}$ sau cùng như
sau:

\begin{algo}Tính $P^{left}$

\begin{enumerate}
\item Đặt $P^{left}(head) = p(head)$
\item Đặt $P^{left}(W) = 0$ với mọi $W$ còn lại.
\item Duyệt lần lượt các nút $W$ theo thứ tự từ trái sang phải, tính
  theo vị trí bắt đầu của $W$ trong câu.
  Với mỗi $W$ tìm được, cộng thêm $p(W'/W)P^{left}(W)$ vào $P^{left}(W')\quad
  W' \in R(W)$
\end{enumerate}
\end{algo}

\begin{algo}Tính $P^{right}$
  
\begin{enumerate}
\item Đặt $P^{right}(tail) = p(tail)$
\item Đặt $P^{right}(W) = 0$ với mọi nút còn lại.
\item Duyệt tất cả các nút $W$ từ phải sang trái, tính theo vị trí kết
  thúc của $W$ trong câu. Với $W' \in L(W)$:
  \begin{enumerate}
  \item Cộng thêm $p(W/W')P^{right}(W)$ vào $P^{right}(W')$
  \item Tính fractional count cho $W'$ theo công thức
    $\displaystyle\frac{P^{left}(W')p(W/W')P^{right}(W)}{P^{left}(tail)}$
  \end{enumerate}
\end{enumerate}
\end{algo}

Với cách tính này, ta có thể dùng lưới từ cùng với 2-gram. Để tính
3-gram, ta dùng lưới 2-từ. 

\subsection{Tính n-gram}

N-gram được tính như thông thường. Sử dụng backoff chuẩn. Thư viện
SRILM được dùng để tính n-gram.

\chapter{Kết luận}
\label{cha:conclusion}

\section{Hạn chế}
\begin{itemize}
\item Không sử dụng những thông tin cấp cao hơn.
\end{itemize}

\section{Hướng phát triển}

Viết lại từ đầu :-D


\begin{thebibliography}{}
\bibitem{Ravishankar}Mosur K. Ravishankar. Efficient Algorithms for
  Speech Recognition, PhD thesis, 1996.
\bibitem{Oflazer}Kemal Oflazer. Error-tolerant Finite State
  Recognition with Applications to Morphological Analysis and Spelling
  Correction, 1996.
\bibitem{LAH}Le An Ha. A method for word segmentation in
  Vietnamese.
\bibitem{Chang}Chao-Huang Chang. A New Approach for
  Automatic Chinese Spelling Correction. 
\bibitem{Sproat}Richard Sproat, William Gale, Chilin Shih, Nancy
  Chang. A Stochastic Finite-State Word-Segmentation Algorithm for
  Chinese.
\bibitem{Chunyu}Chunyu Kit, Zhiming Xu, Jonathan
  J. Webster. Integrating Ngram Model and Case-based Learning For
  Chinese Word Segmentation.
\bibitem{softcount}Xianping Ge, Wanda Pratt,
  Padhraic Smyth. Discovering Chinese Words from Unsegmented Text.
\bibitem{text-tiling}Jianfeng Gao, Hai-Feng Wang, Mingjing Li, Kai-Fu
  Lee. A Unified Approach to Statistical Language Modeling for
  Chinese.
\bibitem{}Nianwen Xue.Chinese Word Segmentation as Character Tagging.
\bibitem{self-supervised}Fuchun Peng and Dale Schuurmans. Self-Supervised Chinese
  Word Segmentation.
\bibitem{phonetex}Victoria J. Hodge, Jim Austin, An Evaluation of
  Phonetic Spell Checkers
\bibitem{soundex}K. Kukich. Techniques for Automatically Correcting
  Words in Text, 1992.
\bibitem{}Cláudio L. Lucchesi and Tomasz Kowaltowski. Applications of
  Finite Automata Representing Large Vocabularies.
\bibitem{}Bo-Hyun Yun, Min-Jeung Cho, Hae-Chang Rim. Segmenting Korean
  Compound Nouns using Statistical Information and a Preference Rule.
\bibitem{}Roger I. W. Spooner and Alistair D. N. Edwards. User
  Modelling for Error Recovery: A Spelling Checker for Dyslexic Users
\bibitem{}Theppitak Karoonboonyanan, Virach Sornlertlamvanich,
  Surapant Meknavin. A Thai Soundex System for Spelling Correction.
\bibitem{}Justin Zobel and Philip Dart. Finding Approximate Matches in
  Large Lexicons.
\bibitem{}Sun Maosong, Shen Dayang, Huang Changning. CSeg\&Tag1.0: A
  Practical Word Segmenter and POS Tagger for Chinese Texts.
\bibitem{wordseg}Đinh Điền, Hoàng Kiếm, Nguyễn Văn Toàn. Vietnamese Word
  Segmentation.
\bibitem{}Bidyut Baran Chaudhuri. Reversed word dictionary and
  phonetically similar word grouping based spell-checker to Banglaa
  text
\bibitem{}Timothy Gambell, Charles D. Yang. Scope and Limits of
  Statistical Learning in Word Segmentation.
\bibitem{iccc}Andi Wu, George Heidorn, Zixin Jiang, Terence
  Peng. Correction of Erroneous Characters in Chinese Sentence
  Analysis
\bibitem{}Fuchun Peng, Xiangji Huang, Dale Schuurmans, Shaojun
  Wang. Text Clasification in Asian Languages without Word
  Segmentation. 
\bibitem{}Yalin Wang, Ihsin T. Phillips, Robert
  Haralick. Statistical-based Approach to Word Segmentation.
\bibitem{}Combining Syntactical And Statistical Language Constraints
  in Context-dependent Language Models for Interactive Speech Applications.
\end{thebibliography}

\end{document}
Giả sử câu S có $n$ tiếng $c_1,c_2,\ldots,c_n$, có $|S|$ cách tách từ
$S_1,S_2,\ldots,S_{|S|}$ và cách tách từ  $S_i$ được
tách thành $|S_i|$ từ $W_{i_1},W_{i_2},\ldots,W_{i_{|S_i|}}$ với 
$W_i$ là một từ xác định bắt đầu ở tiếng thứ $i$ chứa $|W_i|$ tiếng, và
$i_j$ là vị trí từ thứ $j$ trong cách tách câu $i$.

Ta có:
$$p(i)=\sum_{j=1}^{|S_i|}{p(W_{i_j})}$$

Một câu sẽ được tách thành:
$$P(W_i) = P_{i}^{left}p(W_{i})P_{i+|W_i|}^{right}$$

$P_i^{left}$ là xác suất tất cả các tổ hợp từ có thể có từ 
tiếng thứ nhất đến tiếng thứ $i$.

$P_{i}^{right}$ là xác suất tất cả các tổ hợp từ có thể có từ tiếng
thứ  $i$ đến hết câu. 

Dễ thấy, với mỗi từ $C$ trong câu $S$, fractional count W của từ sẽ là
$\displaystyle\frac{P(W)}{\sum_i^{|S|}p(i)}$.

$\displaystyle\sum_i^{|S|}p(i)$ cũng chính là $P_{n+1}^{left}$ theo
định nghĩa $P_i^{left}$.

Ta sẽ dùng quy hoạch động để tính $P_i^{left}$ và $P_i^{right}$

$$
P_i^{left} = \left\{
  \begin{array}{lr}
    1 & i=1\\
    p(W_i) & i=2\\
    \sum_{j=1}^{i-1}p(c_j\ldots c_{i-1})P_j^{left} & i>2
  \end{array}
\right.
$$

$$
p(c_i\ldots c_j) = \left\{
  \begin{array}{ll}
    p(W_i)&\text{nếu } c_i\ldots c_j \text{ tạo thành } W_i\\
    0&\text{ngược lại}
  \end{array}
\right.
$$

Thuật toán tính $P^{left}$ như sau:
\begin{enumerate}
\item Đặt $P_1^{left} = 1$
\item Đặt $P_i^{left} = 0\quad \forall i \in [2\ldots n+1]$
\item Duyệt i từ 1 đến n, tìm tất cả các từ $W_i$ (do tại có thể có
  nhiều từ bắt đầu tại tiếng $i$).
  Với mỗi từ $W_i$ tìm được, cộng thêm $p(W_i)$ vào $P_{i+|W_i|}^{left}$
\end{enumerate}

Thuật toán tương tự được áp dụng để tính $P^{right}$.

Sau khi tính được $P^{left}$ và $P^{right}$, ta có thể tính fractional
count trong câu bằng cách duyệt tất cả các từ có thể có trong câu,
cộng thêm vào $\displaystyle\frac{P(W)}{P_{n+1}^{left}}$ cho từ
$C$. Thực tế, ta sẽ lồng bước này vào trong thuật toán tính
$P^{right}$, vì thuật toán cũng phải duyệt qua tất cả các từ.

Vậy thuật toán tính $P^{right}$ là:
\begin{enumerate}
\item Đặt $P_{n+1}^{right} = 1$
\item Đặt $P_i^{right} = 0\quad \forall i \in [1\ldots n]$
\item Duyệt i từ n+1 đến 1.
  \begin{enumerate}
  \item tìm tất cả các từ $W_j$ sao cho $j+|W_j|=i$.
    Với mỗi từ $W_j$ tìm được, cộng thêm $p(W_j)$ vào
    $P_j^{right}$
  \item Tính fractional count cho tất cả các từ $W_i$ ($i \le n$)
  \end{enumerate}
\end{enumerate}


Ví dụ: câu ``học sinh học sinh học'' có 8 cách tách từ
\begin{verbatim}
học-sinh học-sinh học
học-sinh học sinh học
học-sinh học sinh-học
học sinh học sinh học
học sinh học sinh-học
học sinh học-sinh học
học sinh-học sinh học
học sinh-học sinh-học
\end{verbatim}
\def\Zhs{\text{học-sinh}}
\def\Zsh{\text{sinh-học}}
\def\Zh{\text{học}}
\def\Zs{\text{sinh}}
Ta có
$$
\begin{array}{rl}
P_1^{left}(\Zhs) &= p(\Zhs_1/\phi)\\
P_1^{left}(\Zh) &= p(\Zh_1/\phi)\\
P_2^{left}(\Zs) &= p(\Zs_2/\Zh_1)P_1^{left}(\Zh)\\
 &=p(\Zs_2/\Zh_1)p(\Zh_1/\phi)\\
P_2^{left}(\Zsh) &= p(\Zsh_2/\Zh_1)P_1^{left}(\Zh)\\
 &=p(\Zsh_2/\Zh_1)p(\Zh_1/\phi)\\
P_3^{left}(\Zhs) &= p(\Zhs_3/\Zhs_1)P_1^{left}(\Zhs)+p(\Zhs_3/\Zs_2)P_2^{left}(\Zs)\\
 &=p(\Zhs_3/\Zhs_1)p(\Zhs_1/\phi)+p(\Zhs_3/\Zs_2)p(\Zs_2/\Zh_1)p(\Zh_1/\phi)\\
 &=p(\Zhs_1,\Zhs_3)+p(\Zh_1,\Zs_2,\Zhs_3)\\
P_3^{left}(\Zh) &= p(\Zh_3/\Zs_2)P_2^{left}(\Zs)+p(\Zh_3/\Zhs_1)P_1^{left}(\Zhs)\\
 &=p(\Zh_3/\Zs_2)p(\Zs_2/\Zh_1)p(\Zh_1/\phi)+p(\Zh_3/\Zhs_1)p(\Zhs_1/\phi)\\
 &=p(\Zh_1,\Zs_2,\Zh_3)+p(\Zhs_1,\Zh_3)\\
\end{array}
$$

$$
\begin{array}{rl}
P_5^{right}(\Zh) &= p(\Phi/\Zh_5)\\
P_4^{right}(\Zsh) &= p(\Phi/\Zsh_4)\\
P_4^{right}(\Zs) &= p(\Zh_5/\Zs_4)P_5^{right}(\Zh)\\
 &=p(\Zh_5/\Zs_4)p(\Phi/\Zh_5)\\
P_3^{right}(\Zh) &= p(\Zsh_4/\Zh_3)P_4^{right}(\Zsh)+p(\Zs_4/\Zh_3)P_4^{right}(\Zs)\\
 &=p(\Zsh_4/\Zh_3)p(\Phi/\Zsh_4)+p(\Zs_4/\Zh_3)(\Zh_5/\Zs_4)p(\Phi/\Zh_5)\\
 &=p(\Zh_3,\Zsh_4)+p(\Zh_3,\Zs_4,\Zh_5)\\
P_3^{right}(\Zhs) &= p(\Zh_5/\Zhs_3)P_5^{right}(\Zh)\\
 &=p(\Zh_5/\Zhs_3)p(\Phi/\Zh_5)\\
 &=p(\Zhs_3,\Zh_5)\\
\end{array}
$$

$$
\begin{array}{rl}
P_3(\Zhs) &= P_3^{left}(\Zhs)P_3^{right}(\Zhs)\\
 &=[p(\Zhs_1,\Zhs_3)+p(\Zh_1,\Zs_2,\Zhs_3)]p(\Zhs_3,\Zh_5)\\
 &=p(\Zhs_1,\Zhs_3)p(\Zhs_3,\Zh_5)+p(\Zh_1,\Zs_2,\Zhs_3)p(\Zhs_3,\Zh_5)\\
 &=p(\Zhs_1,\Zhs_3,\Zh_5)+p(\Zh_1,\Zs_2,\Zhs_3,\Zh_5)
\end{array}
$$
